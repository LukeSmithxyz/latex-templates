%This code is based on some written by Stack Exchange member Scott H.

\documentclass[10pt]{article}

\usepackage{xstring}
\usepackage{tikz}
\usetikzlibrary{calc}
\usepackage{xparse}
\usepackage[margin=1in]{geometry}

\input{random.tex}
\newcount\randomnum
\ExplSyntaxOn

\seq_new:N \g_my_items_seq
\seq_new:N \l_my_tmp_items_seq
\seq_new:N \g_my_randomized_seq
\int_new:N \l_tmp_int
\msg_new:nnnn {bingo} {Too~few~items!} {Provide~at~least~24~items!}{}

\cs_generate_variant:Nn \seq_item:Nn {Nx}
\cs_generate_variant:Nn \seq_remove_all:Nn {Nx}

\NewDocumentCommand {\myItems} {m}
    {
      \seq_clear:N \g_my_items_seq % clear item list 
      \seq_gset_split:Nnn \g_my_items_seq {;} {#1} % put item list in seq
      \int_compare:nNnT {\seq_count:N \g_my_items_seq} < {24} {\msg_error:nn {bingo} {Too~few~items!}} % check whether there are enough items
    }

\NewDocumentCommand{\setItems}{}
{
\seq_set_eq:NN \l_my_tmp_items_seq \g_my_items_seq % put in temp seq so that multiple cards can be produced
\prg_replicate:nn {24} %generate random list of 24 items
    {
        \int_set:Nn \l_tmp_int {\seq_count:N \l_my_tmp_items_seq}% set current length of list
        \setrannum{\randomnum}{1}{\int_use:N \l_tmp_int} % choose random num up to length of seq
        \seq_put_right:Nx \g_my_randomized_seq {\seq_item:Nn \l_my_tmp_items_seq {\the\randomnum}}% grab corresponding item and put in tmp seq
        \seq_remove_all:Nx \l_my_tmp_items_seq {\seq_item:Nn \l_my_tmp_items_seq {\the\randomnum}}%delete that item from temp seq
    }
\seq_clear:N \l_my_tmp_items_seq %clear temp seq when done
}

\NewDocumentCommand {\NodeText}{}
    {
        \seq_gpop_right:NN \g_my_randomized_seq \l_tmpa_tl %pop item from randomized seq into token list
        \tl_use:N \l_tmpa_tl %use that item.
    }


\ExplSyntaxOff

\def\NumOfColumns{5}%
\def\Sequence{1, 2, 3, 4, 5}%



\title{Chomsky BINGO!}

\begin{document}

\pagenumbering{gobble}

\newcommand{\Size}{3.25cm}
\tikzset{Square/.style={
inner sep=0pt,
text width=\Size, 
minimum size=\Size,
draw=black,
align=center,
}
}

{\centering\huge Chomsky BINGO!\hrule}
\vspace{.2in}


This BINGO card is valid for the entire duration of Chomsky's stay at the University of Arizona. First one to get BINGO wins\ldots idk what, it's just a game. If you actually yell out ``BINGO!'' during one of his talks maybe I actually will get you something.

\vspace{.5in}

%\myItems{this;will;produce;an;error;because;there;aren't;enough;items}

\myItems{comparing computational linguistics to behaviorism;
kittens and rocks;
black people vs. orangutans anecdote;
corpus linguistics as ``butterfly-collecting'';
EPP is ``well established empirical fact'' (cue laugh track);
Can eagles that fly swim?;
``language `grows' or `maturates' in the mind'';
That sweater? Even in Arizona?;
reference to Port Royal grammar;
``the dogma that language is for communication'';
name-dropping Descartes with questionable relevance;
speculations about a ``Martian scientist'';
steam rises and rocks fall to their ``natural place'';
``Galileo's demolition of the mechanistic world'';
Behaviorism like a ``science of meter-reading'';
``allow one's self to be puzzled'';
some unfounded claim is ``conceptually true'';
irrelevant political aside (bonus points for Trump);
miss the point of a question (basically a free space);
``structural distance'' wins over ``linear distance'';
(repeating anything on this card more than once);
dismissing a whole field/framework as ``not serious'';
construing some new idea he has as being consensus;
extolling the virtues of ignoring data}


\setItems

\begin{tikzpicture}[draw=black, thick, x=\Size,y=\Size]
\foreach \row in \Sequence{%
    \foreach \col in \Sequence {%
        \pgfmathtruncatemacro{\value}{\col+\NumOfColumns*(\row-1)}
        \pgfmathsetmacro{\ColRowProduce}{\col*\row}
        \IfEq{\ColRowProduce}{9}{% If is center square
            \node [Square] at ($(\col,-\row)-(0.5,0.5)$) {\large Characteristic hand-movements\\ (\textbf{FREE})};
        }{
            \node [Square] at ($(\col,-\row)-(0.5,0.5)$) {\large \NodeText};
        }
    }
}    
\end{tikzpicture}

%\setItems

%\begin{tikzpicture}[draw=black, thick, x=\Size,y=\Size]
%\foreach \row in \Sequence{%
%    \foreach \col in \Sequence {%
%        \pgfmathtruncatemacro{\value}{\col+\NumOfColumns*(\row-1)}
%        \pgfmathsetmacro{\ColRowProduce}{\col*\row}
%        \IfEq{\ColRowProduce}{9}{% If is center square
%            \node [] at ($(\col,-\row)-(0.5,0.5)$) {FREE};
%        }{
%            \node [Square] at ($(\col,-\row)-(0.5,0.5)$) {\large \NodeText};
%        }
%    }
%}    
%\end{tikzpicture}                 
\end{document}
